\chapter{Résultats}

% ####################################################################################################

\section{Automatisation du processus de test : scripts Bash}

\paragraph{}
Pour tester notre programme, nous avons sollicité le doctorant Quang Chieu Ta, dont le sujet de thèse
porte sur le problème mathématique dont il est question dans ce projet de \og Sciences de la Décision \fg{}.
Celui-ci nous a fourni 160 instances, réparties en 16 classes contenant 10 instances chacune :
\begin{itemize}
\item[>>] Instance à 2 machines et ...
    \begin{itemize}
    \item ... 10 jobs
    \item ... 20 jobs
    \item ... 30 jobs
    \item ... 50 jobs
    \item ... 70 jobs
    \item ... 100 jobs
    \item ... 150 jobs
    \item ... 200 jobs
    \item ... 250 jobs
    \item ... 300 jobs
    \end{itemize}
\item[>>] Instance à 4 machines et ...
    \begin{itemize}
    \item ... 10 jobs
    \item ... 30 jobs
    \item ... 50 jobs
    \item ... 100 jobs
    \item ... 150 jobs
    \item ... 200 jobs
    \end{itemize}
\end{itemize}

\paragraph{}
Afin de faciliter la phase de réalisation des test, nous avons utilisé deux scripts Bash :
\begin{enumerate}
\item \texttt{call.sh} : script utile à l'invocation du programme de manière automatisée
\item \texttt{compare.sh} : script utile à la comparaison des solutions retournées après exécution du programme
\end{enumerate}
\paragraph{}
Ces deux scripts sont disponibles dans le dossier \texttt{scripts/} fourni avec ce rapport.

\newpage

\subsection{Automatisation de l'invocation de l'exécutable sur plusieurs instances}

\paragraph{}
Le script \texttt{call.sh} prend en entrée les paramètres suivants :
\begin{enumerate}
\item le chemin vers le fichier exécutable correspondant au programme ;
\item le chemin vers un dossier \texttt{instances/} contenant des instances ;
\item le chemin vers un dossier \texttt{solutions/} (existant ou à créer) qui contiendra les solutions ;
\item le nombre de mutations à effectuer à chaque itération (optionnel).
\end{enumerate}
\paragraph{}
Le script invoque automatiquement le programme sur toutes les instances du dossier \texttt{instances/} et les écrit dans le dossier \texttt{solutions/}.

\subsection{Automatisation de la comparaison des résultats obtenus}

\paragraph{}
Le script \texttt{compare.sh} prend en entrée les paramètres suivants :
\begin{enumerate}
\item le chemin vers un dossier \texttt{solutions\_1/} contenant des solutions ;
\item le chemin vers un dossier \texttt{solutions\_2/} contenant des solutions ;
\end{enumerate}
\paragraph{}
Le script recherche tous les couples de fichiers de solutions associés à une même instance :
\begin{itemize}
\item l'un des deux fichiers est situé dans le dossier \texttt{solutions\_1/} ;
\item l'autre fichier est situé dans le dossier \texttt{solutions\_2/} ;
\item les deux fichiers portent le même nom \texttt{name.txt}.
\end{itemize}
Si les deux fichiers sont différents, \texttt{name.txt} est affiché à l'écran.
\paragraph{}
À la fin, le programme affiche le nombre total de couples trouvés, le nombre de couples constitués de deux solutions différentes,
avec leur proportion (en pourcentage).

\newpage

% ####################################################################################################

\section{Résultats de l'heuristique}

\paragraph{}
Nous avons utilisé le script \texttt{call.sh} sur toutes les instances dont nous disposons, regroupées dans le dossier \texttt{files/instances/}.
Dans le dossier \texttt{files/solutions/} fourni avec ce rapport, les solutions obtenues avec différentes valeurs du paramètre \og nombre de mutations \fg{} pour toutes les instances sont fournies. Ces solutions sont regroupées en sous-dossiers de la manière suivante :
\begin{center}
\texttt{S\_(valeur du paramètre : nombre de mutations)\_(lettre de \emph{a} à \emph{d})}
\end{center}
\begin{itemize}
\item Si la valeur du paramètre \og nombre de mutations \fg{} est vide, alors le script a été appelé sans préciser de valeur :
pour chaque instance, le nombre de mutations par itération est égal au nombre de jobs dans l'instance.
\item Les lettres servent à différencier plusieurs appels au script avec la même valeur pour le paramètre \og nombre de mutations \fg{}.
\end{itemize}

\paragraph{}
Des fichiers textes de comparaison des solutions obtenues sont également disponibles dans le dossier \texttt{files/solutions/compare/}.
Leur contenu est une copie de ce qu'affiche le script \texttt{compare.sh} (redirection de la sortie standard).
Leur nom est de la forme suivante :
\begin{center}
\texttt{compare\_(nom dossier solutions 1)\_(nom dossier solutions 2).txt}
\end{center}

\subsection{Variabilité des solutions générées}

\paragraph{}
L'heuristique implémentée est extrêmement instable : en appelant deux fois de suite le programme sur la même instance, les probabilités
d'obtenir deux solutions différentes sont élevées. Ceci est dû aux choix aléatoires de solutions dans le voisinage de la solution mère
dans l'algorithme de descente locale.

\paragraph{}
Pour plus de détails, veuillez consulter les données des fichiers de comparaison, pour une même valeur du paramètre \og nombre de mutations \fg{}
et des lettres différentes.

\subsection{Influence du paramètre : nombre de mutations}

\paragraph{}
Si on augmente la valeur du paramètre \og nombre de mutations \fg{} :
\begin{enumerate}
\item Les chances d'obtenir des solutions proches d'une solution optimale sont de plus en plus grandes. (La valeur de la somme des retards diminue.)
\item Le temps d'exécution augmente :
    \begin{itemize}
    \item raisonnablement pour des instances avec peu de jobs (jusqu'à 100 jobs, moins de 5 secondes environ) ;
    \item déraisonnablement pour des instances avec beaucoup de jobs (au-delà de 150 jobs, 1 à 10 minutes environ).
    \end{itemize}
\item Le phénomène d'instabilité est accru.
\end{enumerate}

\paragraph{}
Pour plus de détails, veuillez consulter les données des fichiers de comparaison, pour des valeurs différentes du paramètre \og nombre de mutations \fg{} (quelles que soient les lettres).

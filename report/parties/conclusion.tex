\chapter{Conclusion}

\paragraph{}
L'implémentation de l'heuristique a été réussie.
Selon la valeur attribuée au nombre de mutations par itération, le temps d'exécution est plus ou moins acceptable.
Du fait de l'exploration pseudo-aléatoire du voisinage d'une solution donnée, l'algorithme est extrêmement instable.
Une piste envisageable pour limiter cette instabilité serait la mise en place d'un tabou : en maintenant à jour
un historique (même partiel) de l'exploration effectuée, il deviendrait alors possible de s'extraire des minima locaux,
qui ne sont généralement pas des solutions optimales (minima globaux).

\vfill

\section{Conclusion de Pierre-Antoine}

\paragraph{}
Ce projet était la suite logique du mini-projet de Gestion des Flux et de la Production.
J'ai apprécié le fait de pouvoir travailler sur une thématique très intéressante, de manière prolongée dans le temps,
pour mieux appréhender le processus de réflexion associé à la recherche d'une solution heuristique (première approche simple, puis optimisation).
Par ailleurs, les compétences d'Armand et les miennes se sont révélées être complémentaires, une fois encore.

\section{Conclusion de Armand}

\paragraph{}
Lors de ce projet, j'ai pu apprendre à utiliser de nouvelles méthodes de résolution en particulier l'utilisation de mutations pour une recherche de minimum local et voir le résultat sur de nombreuses instances. De plus travailler de nouveau avec Pierre-Antoine fut à nouveau une expérience enrichissante.

\vfill

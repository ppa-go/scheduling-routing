\documentclass[twoside,UTF8,final]{EPURapport}

\thedocument{Rapport - Projet de Sciences de la Décision}%
{Heuristique pour un problème d'ordonnancement et de tournées}%
{Heuristique pour un problème d'ordonnancement et de tournées}

\grade{Département Informatique \\ 4\ieme{} année \\ 2013 - 2014}

\authors{%
	\category{Étudiants}{%
		\name{Pierre-Antoine MORIN} \mail{pierre-antoine.morin@etu.univ-tours.fr}
		\name{Armand RENAUDEAU} \mail{armand.renaudeau@etu.univ-tours.fr}
	}
	\details{DI4 2013 - 2014}
}

\supervisors{%
	\category{Encadrant}{%
		\name{M. Jean-Charles BILLAUT} \mail{jean-charles.billaut@univ-tours.fr}
		% \details{(des détails)}
	}
    \details{%
        Université François-Rabelais, Tours
        % (autres détails) \\
        % \href{lien complet}{\texttt{texte affiché pour le lien}}
    }
}

\abstracts{Implémentation en langage C d'une heuristique pour un problème d'ordonnancement et de tournées. Le principe est de regrouper les jobs en groupes, pour lesquels l'algorithme de NEH fournit l'ordonnancement et la recherche du plus proche voisin fournit la tournée. Une descente locale est réalisée à partir d'une solution initiale.}%
{ordonnancement, tournée, heuristique, descente locale}%
{C language implementation of a heuristic for a scheduling and routing problem. The idea is to gather jobs in groups, for which the NEH algorithm provides the scheduling order and the research of the nearest neighbor provides the routing order. A local descent is made from an initial solution.}%
{scheduling, routing, heuristic, local descent}

% \entitylogo{logos/(nom du fichier, sans extension)}

% \nolistoffigures
% \nolistoftables

% \usepackage[options du package]{nom du package}

% Pierre-Antoine Morin
% environnement_algorithmes_fr.tex

\usepackage[boxed]{algorithm}
\usepackage{algorithmic}

\algsetup{linenodelimiter=.}
\algsetup{indent=2em}

\renewcommand{\algorithmicrequire}{\textbf{Préconditions}}
\renewcommand{\algorithmicensure}{\textbf{Postconditions}}
\renewcommand{\algorithmicend}{\textbf{Fin}}
\renewcommand{\algorithmicif}{\textbf{Si}}
\renewcommand{\algorithmicthen}{\textbf{Alors}}
\renewcommand{\algorithmicelse}{\textbf{Sinon}}
\renewcommand{\algorithmicfor}{\textbf{Pour}}
\renewcommand{\algorithmicto}{\textbf{à}}
\renewcommand{\algorithmicforall}{\textbf{Pour tout}}
\renewcommand{\algorithmicdo}{\textbf{Faire}}
\renewcommand{\algorithmicwhile}{\textbf{Tant que}}
\renewcommand{\algorithmicloop}{\textbf{Boucle}}
\renewcommand{\algorithmicrepeat}{\textbf{Répéter}}
\renewcommand{\algorithmicuntil}{\textbf{Jusqu'à}}
\renewcommand{\algorithmicprint}{\textbf{afficher}}
\renewcommand{\algorithmicreturn}{\textbf{retourner}}
\renewcommand{\algorithmictrue}{\textbf{vrai}}
\renewcommand{\algorithmicfalse}{\textbf{faux}}
\renewcommand{\algorithmiccomment}[1]{/* \textit{#1} */}

\floatname{algorithm}{Algorithme}


\graphicspath{{images/}}

\begin{document}

\chapter{Introduction}

% Le texte de l'introduction


\chapter{Présentation du problème}

% ####################################################################################################

\section{Description générale}

\paragraph{}
Une entreprise est spécialisée dans la fabrication de produits. Le procédé de fabrication est unique, il correspond à la gamme suivante :
tous les jobs doivent être traités successivement par toutes les machines, dans le même ordre.
L’atelier de production (l’usine) est organisé en flowshop.

\paragraph{}
Après fabrication, l’entreprise assure la livraison de la marchandise jusqu’à un site choisi par le client. Pour ce faire, l’entreprise dispose d’un unique véhicule, de capacité supposée infinie. (au cours d’une tournée de livraison, on peut transporter une quantité de marchandise aussi grande que voulue)

\paragraph{}
Le carnet de commandes est constitué de tâches (jobs), pour lesquelles on connaît :
\begin{itemize}
\item la durée de passage sur chacune des machines ;
\item la date à laquelle le client souhaite être livré ;
\item le site sur lequel livrer la marchandise.
\end{itemize}

\paragraph{}
Les temps de trajet d’un site à un autre, y compris l’usine, sont tous connus.

\paragraph{}
Le retard d’un job est défini ainsi :
\begin{itemize}
\item Si la marchandise est livrée sur le site choisi par le client avant la date demandée, alors le retard est nul.
\item En revanche, si cette date limite est dépassée, le retard est égal à la différence entre la date de livraison effective et la date de livraison demandée.
\end{itemize}

\paragraph{}
\textbf{Quelle organisation en atelier et quelles tournées de livraison effectuer afin de minimiser la somme des retards ?}

\newpage

% ####################################################################################################

\section{Formulation mathématique}

\subsection*{Données caractérisant la taille d'une instance}

\begin{tabular}{cl}
$ n $ & Nombre de jobs $ J_{j} $ = Nombre de sites $ S_{a} $ en plus de l'usine $ U = S_{0} $ \\
$ m $ & Nombre de machines $ M_{i} $ \\
\end{tabular}

\subsection*{Données caractérisant un job $ J_{j} $}

\begin{tabular}{ccl}
$ \begin{array}{l} \forall i = 1 \, .. \, m \\ \forall j = 1 \, .. \, n \end{array} $ & $ p_{i \, j}$ & \textit{Processing time} du job $ J_{j} $ sur la machine $ M_{i} $ \\
\\
$ \forall j = 1 \, .. \, n $ & $ d_{j} $ & \textit{Due date} du job $ J_{j} $ \\
\end{tabular}

\subsection*{Données caractérisant les durées de transport de marchandises}

\begin{tabular}{ccl}
$ \begin{array}{l} \forall a = 0 \, .. \, n \\ \forall b = 0 \, .. \, n \end{array} $ & $ t_{a \, b} $ & Durée de voyage pour aller du site $ S_{a} $ au site $ S_{b} $ \\
\end{tabular}

\subsection*{Remarque importante}

Le job $ J_{j} $ doit être livré sur le site $ S_{j} $.
On suppose que tous les sites sont différents.
En pratique, si les jobs $ J_{a} $ et $ J_{b} $ doivent être livrés sur le même site, on posera : $ t_{a \, b} = t_{b \, a} = 0 $.

\subsection*{Format des données}

\begin{center}

\begin{tabular}{ccc}

\begin{tabular}{c|ccccc|}
$ \mathcal{P} $ & $ J_{1} $ & $ \cdots $ & $ J_{j} $ & $ \cdots $ & $ J_{n} $ \\
\hline
$ \mathcal{M}_{1} $ & $ p_{1 \, 1} $ & $ \cdots $ & $ p_{1 \, j} $ & $ \cdots $ & $ p_{1 \, n} $ \\
$ \vdots $ & $ \vdots $ & & $ \vdots $ & & $ \vdots $ \\
$ \mathcal{M}_{i} $ & $ p_{i \, 1} $ & $ \cdots $ & $ p_{i \, j} $ & $ \cdots $ & $ p_{i \, n} $ \\
$ \vdots $ & $ \vdots $ & & $ \vdots $ & & $ \vdots $ \\
$ \mathcal{M}_{m} $ & $ p_{m \, 1} $ & $ \cdots $ & $ p_{m \, j} $ & $ \cdots $ & $ p_{m \, n} $ \\
\hline
$ \mathcal{D} $ & $ d_{1} $ & $ \cdots $ & $ d_{j} $ & $ \cdots $ & $ d_{n} $ \\
\hline
\end{tabular}

& &

\begin{tabular}{c|cccccc|}
$ \mathcal{T} $ & $ U $ & $ S_{1} $ & $ \cdots $ & $ S_{b} $ & $ \cdots $ & $ S_{n} $ \\
\hline
$ U $ & \textbf{0} & $ t_{0 \, 1} $ & $ \cdots $ & $ t_{0 \, b} $ & $ \cdots $ & $ t_{0 \, n} $ \\
$ S_{1} $ & $ t_{1 \, 0} $ & \textbf{0} & $ \cdots $ & $ t_{1 \, b} $ & $ \cdots $ & $ t_{1 \, n} $ \\
$ \vdots $ & $ \vdots $ & $ \vdots $ & & $ \vdots $ & & $ \vdots $ \\
$ S_{a} $ & $ t_{a \, 0} $ & $ t_{a \, 1} $ & $ \cdots $ & $ t_{a \, b} $ & $ \cdots $ & $ t_{a \, n} $ \\
$ \vdots $ & $ \vdots $ & $ \vdots $ & & $ \vdots $ & & $ \vdots $ \\
$ S_{n} $ & $ t_{n \, 0} $ & $ t_{n \, 1} $ & $ \cdots $ & $ t_{n \, b} $ & $ \cdots $ & \textbf{0} \\
\hline
\end{tabular}

\end{tabular}

\end{center}

\subsection*{Objectif}

\begin{center}
\[ minimiser \left( \sum_{j = 1}^{n}{T_j} \right) \]
où la variable $ T_j $ représente le retard associé à la livraison du job $ J_j $. ($ T_j \geq 0 $)
\end{center}
\chapter{Cahier des charges}

% ####################################################################################################

%
\chapter{Gestion de projet}

% ####################################################################################################

%
\chapter{Algorithme utilisé pour l'heuristique}

% ####################################################################################################

\section{Recherche d'une solution initiale}

%

% ####################################################################################################

\section{Voisinage d'une solution}

%

% ####################################################################################################

\section{Descente locale}

%
\chapter{Implémentation}


% ####################################################################################################
	\paragraph{}
	Comme spécifié dans le cahier des charges, notre programme a été implémenté en utilisant le langage C. Pour fonctionner, le programme a besoin d'un fichier contenant différentes données qui sont présentées d'une manière bien précise. À la fin de l'exécution, le programme crée un fichier de sortie contenant des données également présentées d'une manière bien précise. Dans un premier temps, nous allons détailler le format des fichiers d'entrée et de sortie, puis, dans un deuxième temps, la documentation qui présente l'architecture du projet.
	
	\paragraph{}
	Les fichiers d'en-têtes sont disponibles dans le dossier \texttt{program/headers/} et les fichiers sources sont disponibles dans le dossier \texttt{program/sources/}. La création du programme compilé peut se faire à l'aide de la commande \textit{make} grâce au \textit{Makefile} qui est présent dans le dossier \texttt{program/}. L'executable alors généré est présent dans le dossier \texttt{program/bin/} et se nomme \textit{program.exe}.
%

% ####################################################################################################
	\section{Format des fichiers}
		\subsection{Fichiers d'entrée}
	
	\paragraph{}
	Comme présenté précédemment, les fichiers d'entrée sont présentés d'une manière bien précise. Concernant le nom des fichiers, ils sont tous nommées de la même manière, à savoir \texttt{I\_(nombre de machines)\_(nombre de jobs)\_(numero d'instance)}.txt. Voici comment sont présentés ces fichiers. 
	\begin{flushleft}
	(nombre de machines)(tabulation)(nombre de jobs) \\
	(durées sur la première machine, séparées par des tabulations) \\
	... \\
	(durées sur la dernière machine, séparées par des tabulations) \\
	(dates dues, séparées par des tabulations) \\
	(distances de l'usine aux autres sites, séparées par des tabulations) \\
	(distances du premier site aux autres sites, séparées par des tabulations) \\
	... \\
	(distances du dernier site aux autres sites, séparées par des tabulations)
	\end{flushleft}
%

% ####################################################################################################
	\subsection{Fichiers de sortie}
	
	\paragraph{}
	Les fichiers de sorties sont tous nommés de la même manière, à savoir \texttt{S\_(nombre de machines)\_(nombre de jobs)\_(numero d'instance).txt}. Voici comment sont présentés ces fichiers. 
	\begin{flushleft}
	(somme des retards) \\
	(temps CPU en millisecondes) \\
	(numéros des jobs selon la séquence d'ordonnancement, séparés par des tabulations) \\
	(nombre de tournées) \\
	(nombres de sites par tournée, séparés par des tabulations) \\
	(numéros des sites de la première tournée selon l'ordre de parcours, séparés par des tabulations) \\
	... \\
	(numéros des sites de la première tournée selon l'ordre de parcours, séparés par des tabulations) \\
	\end{flushleft}
%

% ####################################################################################################
	\section{Documentation}
	
	\paragraph{}
	Le projet a été documenté grâce à l'outils doxygen. Pour visualiser la documentation, il faut se rendre dans le dossier du projet et ouvrir le fichier index.html qui se trouve dans le dossier \texttt{documentation/ html/}

% ####################################################################################################	
	\section{Utilisation du programme}
	
	\paragraph{}
	Pour fonctionner, le programme que nous avons réalisé a besoin de plusieurs paramètres :
	\begin{enumerate}
		\item Le fichier contanant l'instance à traiter.
		\item Le fichier contenant les résultats après l'execution du programme.
		\item Le nombre de mutuations. Ce paramètre est optionnel, c'est-à-dire que s'il n'est pas indiquer, le nombre de mutations effectuées est égal au nombre de jobs de l'instance.
	\end{enumerate}	
%
\chapter{Résultats}

% ####################################################################################################

\section{Résultats de l'heuristique}

%

% ####################################################################################################

\section{Comparaison des résultats par rapport à une autre heuristique}

%

\chapter{Conclusion}

\paragraph{}
L'implémentation de l'heuristique a été réussie.
Selon la valeur attribuée au nombre de mutations par itération, le temps d'exécution est plus ou moins acceptable.
Du fait de l'exploration pseudo-aléatoire du voisinage d'une solution donnée, l'algorithme est extrêmement instable.
Une piste envisageable pour limiter cette instabilité serait la mise en place d'un tabou : en maintenant à jour
un historique (même partiel) de l'exploration effectuée, il deviendrait alors possible de s'extraire des minima locaux,
qui ne sont généralement pas des solutions optimales (minima globaux).

\vfill

\section{Conclusion de Pierre-Antoine}

\paragraph{}
Ce projet était la suite logique du mini-projet de Gestion des Flux et de la Production.
J'ai apprécié le fait de pouvoir travailler sur une thématique très intéressante, de manière prolongée dans le temps,
pour mieux appréhender le processus de réflexion associé à la recherche d'une solution heuristique (première approche simple, puis optimisation).
Par ailleurs, les compétences d'Armand et les miennes se sont révélées être complémentaires, une fois encore.

\section{Conclusion de Armand}

\paragraph{}
Lors de ce projet, j'ai pu apprendre à utiliser de nouvelles méthodes de résolution en particulier l'utilisation de mutations pour une recherche de minimum local et voir le résultat sur de nombreuses instances. De plus travailler de nouveau avec Pierre-Antoine fut à nouveau une expérience enrichissante.

\vfill


\annexes

\end{document}

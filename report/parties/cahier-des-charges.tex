\chapter{Cahier des charges}

% ####################################################################################################
	\paragraph{}
		Les exigences de notre encadrant pour ce projet sont les suivantes :
\begin{enumerate}
\item Création d'un algorithme d'une heuristique permettant la résolution d'un problème d'ordonnancement avec routing. Cette heuristique propose une solution initiale et à l'aide de mutations effectuées lors de l'ordonnancement, elle essayera d'améliorer cette dernière.
\item Implémenter l'heuristique en langage C.
\item Tester celle-ci sur plusieurs instances.
\item Analyser les résultats de cette heuristique pour voir quelle a été l'efficacité de celle-ci et voir si on ne peut pas essayer de trouver d'autres méthodes de résolution pour améliorer cette heuristique
\end{enumerate} 

	\paragraph{}
	Nous allons maintenant passer à la prochaine partie qui est l'explication de l'alogorithme de l'heuristique.
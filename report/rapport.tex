\documentclass[twoside,UTF8,final]{EPURapport}

\thedocument{Rapport - Projet de Sciences de la Décision}%
{Heuristique pour un problème d'ordonnancement et de tournées}%
{Heuristique pour un problème d'ordonnancement et de tournées}

\grade{Département Informatique \\ 4\ieme{} année \\ 2013 - 2014}

\authors{%
	\category{Étudiants}{%
		\name{Pierre-Antoine MORIN} \mail{pierre-antoine.morin@etu.univ-tours.fr}
		\name{Armand RENAUDEAU} \mail{armand.renaudeau@etu.univ-tours.fr}
	}
	\details{DI4 2013 - 2014}
}

\supervisors{%
	\category{Encadrant}{%
		\name{M. Jean-Charles BILLAUT} \mail{jean-charles.billaut@univ-tours.fr}
		% \details{(des détails)}
	}
    \details{%
        Université François-Rabelais, Tours
        % (autres détails) \\
        % \href{lien complet}{\texttt{texte affiché pour le lien}}
    }
}

\abstracts{Implémentation en langage C d'une heuristique pour un problème d'ordonnancement et de tournées. Le principe est de regrouper les jobs en groupes, pour lesquels l'algorithme de NEH fournit l'ordonnancement et la recherche du plus proche voisin fournit la tournée. Une descente locale est réalisée à partir d'une solution initiale.}%
{ordonnancement, tournée, heuristique, descente locale}%
{C language implementation of a heuristic for a scheduling and routing problem. The idea is to gather jobs in groups, for which the NEH algorithm provides the scheduling order and the research of the nearest neighbor provides the routing order. A local descent is made from an initial solution.}%
{scheduling, routing, heuristic, local descent}

% \entitylogo{logos/(nom du fichier, sans extension)}

% \nolistoffigures
% \nolistoftables

% \usepackage[options du package]{nom du package}

% Pierre-Antoine Morin
% environnement_algorithmes_fr.tex

\usepackage[boxed]{algorithm}
\usepackage{algorithmic}

\algsetup{linenodelimiter=.}
\algsetup{indent=2em}

\renewcommand{\algorithmicrequire}{\textbf{Préconditions}}
\renewcommand{\algorithmicensure}{\textbf{Postconditions}}
\renewcommand{\algorithmicend}{\textbf{Fin}}
\renewcommand{\algorithmicif}{\textbf{Si}}
\renewcommand{\algorithmicthen}{\textbf{Alors}}
\renewcommand{\algorithmicelse}{\textbf{Sinon}}
\renewcommand{\algorithmicfor}{\textbf{Pour}}
\renewcommand{\algorithmicto}{\textbf{à}}
\renewcommand{\algorithmicforall}{\textbf{Pour tout}}
\renewcommand{\algorithmicdo}{\textbf{Faire}}
\renewcommand{\algorithmicwhile}{\textbf{Tant que}}
\renewcommand{\algorithmicloop}{\textbf{Boucle}}
\renewcommand{\algorithmicrepeat}{\textbf{Répéter}}
\renewcommand{\algorithmicuntil}{\textbf{Jusqu'à}}
\renewcommand{\algorithmicprint}{\textbf{afficher}}
\renewcommand{\algorithmicreturn}{\textbf{retourner}}
\renewcommand{\algorithmictrue}{\textbf{vrai}}
\renewcommand{\algorithmicfalse}{\textbf{faux}}
\renewcommand{\algorithmiccomment}[1]{/* \textit{#1} */}

\floatname{algorithm}{Algorithme}


\graphicspath{{images/}}

\begin{document}

\chapter{Introduction}

\paragraph{}
Ce mini-projet de science de la décision est une continuation de notre ancien projet de Gestion de Flux de Production. Le but de celui-ci est de d'améliorer l'heuristique que nous avions mis en place la dernière fois en appliquant de nouvelles méthodes de résolution.

\paragraph{}
Dans le premier chapitre de ce rapport, nous présenterons le problème abordé, à savoir $ F_m $ | $ \sum{T_{j}} $ avec VRP (\textit{Vehicle Routing Problem}).
Ensuite, nous évoquerons les attentes de notre encadrant dans le cahier des charges.
Dans le troisième chapitre, nous aborderons l'algorithme de l'heuristique.
Apres ce chapitre, nous parlerons de comment nous avons implémenté notre heuristique dans un langage de programmation.
Dans le dernier chapitre, nous présenterons les résultats que l'heuristique a produit sur de nombreuses instances ainsi qu'une analyse détaillée de ces résultats
Enfin, nous terminerons par une conclusion personnelle à propos des enjeux de ce projet et de ce que nous en avons appris.


\chapter{Présentation du problème}

% ####################################################################################################

\section{Description générale}

\paragraph{}
Une entreprise est spécialisée dans la fabrication de produits. Le procédé de fabrication est unique, il correspond à la gamme suivante :
tous les jobs doivent être traités successivement par toutes les machines, dans le même ordre.
L’atelier de production (l’usine) est organisé en flowshop.

\paragraph{}
Après fabrication, l’entreprise assure la livraison de la marchandise jusqu’à un site choisi par le client. Pour ce faire, l’entreprise dispose d’un unique véhicule, de capacité supposée infinie. (au cours d’une tournée de livraison, on peut transporter une quantité de marchandise aussi grande que voulue)

\paragraph{}
Le carnet de commandes est constitué de tâches (jobs), pour lesquelles on connaît :
\begin{itemize}
\item la durée de passage sur chacune des machines ;
\item la date à laquelle le client souhaite être livré ;
\item le site sur lequel livrer la marchandise.
\end{itemize}

\paragraph{}
Les temps de trajet d’un site à un autre, y compris l’usine, sont tous connus.

\paragraph{}
Le retard d’un job est défini ainsi :
\begin{itemize}
\item Si la marchandise est livrée sur le site choisi par le client avant la date demandée, alors le retard est nul.
\item En revanche, si cette date limite est dépassée, le retard est égal à la différence entre la date de livraison effective et la date de livraison demandée.
\end{itemize}

\paragraph{}
\textbf{Quelle organisation en atelier et quelles tournées de livraison effectuer afin de minimiser la somme des retards ?}

\newpage

% ####################################################################################################

\section{Formulation mathématique}

\subsection*{Données caractérisant la taille d'une instance}

\begin{tabular}{cl}
$ n $ & Nombre de jobs $ J_{j} $ = Nombre de sites $ S_{a} $ en plus de l'usine $ U = S_{0} $ \\
$ m $ & Nombre de machines $ M_{i} $ \\
\end{tabular}

\subsection*{Données caractérisant un job $ J_{j} $}

\begin{tabular}{ccl}
$ \begin{array}{l} \forall i = 1 \, .. \, m \\ \forall j = 1 \, .. \, n \end{array} $ & $ p_{i \, j}$ & \textit{Processing time} du job $ J_{j} $ sur la machine $ M_{i} $ \\
\\
$ \forall j = 1 \, .. \, n $ & $ d_{j} $ & \textit{Due date} du job $ J_{j} $ \\
\end{tabular}

\subsection*{Données caractérisant les durées de transport de marchandises}

\begin{tabular}{ccl}
$ \begin{array}{l} \forall a = 0 \, .. \, n \\ \forall b = 0 \, .. \, n \end{array} $ & $ t_{a \, b} $ & Durée de voyage pour aller du site $ S_{a} $ au site $ S_{b} $ \\
\end{tabular}

\subsection*{Remarque importante}

Le job $ J_{j} $ doit être livré sur le site $ S_{j} $.
On suppose que tous les sites sont différents.
En pratique, si les jobs $ J_{a} $ et $ J_{b} $ doivent être livrés sur le même site, on posera : $ t_{a \, b} = t_{b \, a} = 0 $.

\subsection*{Format des données}

\begin{center}

\begin{tabular}{ccc}

\begin{tabular}{c|ccccc|}
$ \mathcal{P} $ & $ J_{1} $ & $ \cdots $ & $ J_{j} $ & $ \cdots $ & $ J_{n} $ \\
\hline
$ \mathcal{M}_{1} $ & $ p_{1 \, 1} $ & $ \cdots $ & $ p_{1 \, j} $ & $ \cdots $ & $ p_{1 \, n} $ \\
$ \vdots $ & $ \vdots $ & & $ \vdots $ & & $ \vdots $ \\
$ \mathcal{M}_{i} $ & $ p_{i \, 1} $ & $ \cdots $ & $ p_{i \, j} $ & $ \cdots $ & $ p_{i \, n} $ \\
$ \vdots $ & $ \vdots $ & & $ \vdots $ & & $ \vdots $ \\
$ \mathcal{M}_{m} $ & $ p_{m \, 1} $ & $ \cdots $ & $ p_{m \, j} $ & $ \cdots $ & $ p_{m \, n} $ \\
\hline
$ \mathcal{D} $ & $ d_{1} $ & $ \cdots $ & $ d_{j} $ & $ \cdots $ & $ d_{n} $ \\
\hline
\end{tabular}

& &

\begin{tabular}{c|cccccc|}
$ \mathcal{T} $ & $ U $ & $ S_{1} $ & $ \cdots $ & $ S_{b} $ & $ \cdots $ & $ S_{n} $ \\
\hline
$ U $ & \textbf{0} & $ t_{0 \, 1} $ & $ \cdots $ & $ t_{0 \, b} $ & $ \cdots $ & $ t_{0 \, n} $ \\
$ S_{1} $ & $ t_{1 \, 0} $ & \textbf{0} & $ \cdots $ & $ t_{1 \, b} $ & $ \cdots $ & $ t_{1 \, n} $ \\
$ \vdots $ & $ \vdots $ & $ \vdots $ & & $ \vdots $ & & $ \vdots $ \\
$ S_{a} $ & $ t_{a \, 0} $ & $ t_{a \, 1} $ & $ \cdots $ & $ t_{a \, b} $ & $ \cdots $ & $ t_{a \, n} $ \\
$ \vdots $ & $ \vdots $ & $ \vdots $ & & $ \vdots $ & & $ \vdots $ \\
$ S_{n} $ & $ t_{n \, 0} $ & $ t_{n \, 1} $ & $ \cdots $ & $ t_{n \, b} $ & $ \cdots $ & \textbf{0} \\
\hline
\end{tabular}

\end{tabular}

\end{center}

\subsection*{Objectif}

\begin{center}
\[ minimiser \left( \sum_{j = 1}^{n}{T_j} \right) \]
où la variable $ T_j $ représente le retard associé à la livraison du job $ J_j $. ($ T_j \geq 0 $)
\end{center}
\chapter{Cahier des charges}

% ####################################################################################################
	\paragraph{}
		Les exigences de notre encadrant pour ce projet sont les suivantes :
\begin{enumerate}
\item Création d'un algorithme d'une heuristique permettant la résolution d'un problème d'ordonnancement avec routing. Cette heuristique propose une solution initiale et à l'aide de mutations effectuées lors de l'ordonnancement, elle essayera d'améliorer cette dernière.
\item Implémenter l'heuristique en langage C.
\item Tester celle-ci sur plusieurs instances.
\item Analyser les résultats de cette heuristique pour voir quelle a été l'efficacité de celle-ci et voir si on ne peut pas essayer de trouver d'autres méthodes de résolution pour améliorer cette heuristique
\end{enumerate} 

	\paragraph{}
	Nous allons maintenant passer à la prochaine partie qui est l'explication de l'alogorithme de l'heuristique.
\input{parties/gestion-de-projet}
\chapter{Algorithme utilisé pour l'heuristique}

% ####################################################################################################

\section{Recherche d'une solution initiale}

%

% ####################################################################################################

\section{Voisinage d'une solution}

%

% ####################################################################################################

\section{Descente locale}

%
\chapter{Implémentation}


% ####################################################################################################
	\paragraph{}
	Comme spécifié dans le cahier des charges, notre programme a été implémenté en utilisant le langage C. Pour fonctionner, le programme a besoin d'un fichier contenant différentes données qui sont présentées d'une manière bien précise. À la fin de l'exécution, le programme crée un fichier de sortie contenant des données également présentées d'une manière bien précise. Dans un premier temps, nous allons détailler le format des fichiers d'entrée et de sortie, puis, dans un deuxième temps, la documentation qui présente l'architecture du projet.
%

% ####################################################################################################
	\section{Format des fichiers}
		\subsection{Fichiers d'entrée}
	
	\paragraph{}
	Comme présenté précédemment, les fichiers d'entrée sont présentés d'une manière bien précise. Concernant le nom des fichiers, ils sont tous nommées de la même manière, à savoir \texttt{I\_(nombre de machines)\_(nombre de jobs)\_(numero d'instance)}.txt. Voici comment sont présentés ces fichiers. 
	\begin{flushleft}
	(nombre de machines)(tabulation)(nombre de jobs) \\
	(durées sur la première machine, séparées par des tabulations) \\
	... \\
	(durées sur la dernière machine, séparées par des tabulations) \\
	(dates dues, séparées par des tabulations) \\
	(distances de l'usine aux autres sites, séparées par des tabulations) \\
	(distances du premier site aux autres sites, séparées par des tabulations) \\
	... \\
	(distances du dernier site aux autres sites, séparées par des tabulations)
	\end{flushleft}
%

% ####################################################################################################
	\subsection{Fichiers de sortie}
	
	\paragraph{}
	Les fichiers de sorties sont tous nommés de la même manière, à savoir \texttt{S\_(nombre de machines)\_(nombre de jobs)\_(numero d'instance).txt}. Voici comment sont présentés ces fichiers. 
	\begin{flushleft}
	(somme des retards) \\
	(temps CPU en millisecondes) \\
	(numéros des jobs selon la séquence d'ordonnancement, séparés par des tabulations) \\
	(nombre de tournées) \\
	(nombres de sites par tournée, séparés par des tabulations) \\
	(numéros des sites de la première tournée selon l'ordre de parcours, séparés par des tabulations) \\
	... \\
	(numéros des sites de la première tournée selon l'ordre de parcours, séparés par des tabulations) \\
	\end{flushleft}
%

% ####################################################################################################
	\section{Documentation}
	
	\paragraph{}
	Le projet a été documenté grâce à l'outils doxygen. Pour visualiser la documentation, il faut se rendre dans le dossier du projet et ouvrir le fichier index.html qui se trouve dans le dossier \texttt{Documentation/ html/}
%
\chapter{Résultats}

% ####################################################################################################

\section{Automatisation du processus de test : scripts Bash}

\paragraph{}
Pour tester notre programme, nous avons sollicité le doctorant Quang Chieu Ta, dont le sujet de thèse
porte sur le problème mathématique dont il est question dans ce projet de \og Sciences de la Décision \fg{}.
Celui-ci nous a fourni 160 instances, réparties en 16 classes contenant 10 instances chacune :
\begin{itemize}
\item[>>] Instance à 2 machines et ...
    \begin{itemize}
    \item ... 10 jobs
    \item ... 20 jobs
    \item ... 30 jobs
    \item ... 50 jobs
    \item ... 70 jobs
    \item ... 100 jobs
    \item ... 150 jobs
    \item ... 200 jobs
    \item ... 250 jobs
    \item ... 300 jobs
    \end{itemize}
\item[>>] Instance à 4 machines et ...
    \begin{itemize}
    \item ... 10 jobs
    \item ... 30 jobs
    \item ... 50 jobs
    \item ... 100 jobs
    \item ... 150 jobs
    \item ... 200 jobs
    \end{itemize}
\end{itemize}

\paragraph{}
Afin de faciliter la phase de réalisation des test, nous avons utilisé deux scripts Bash :
\begin{enumerate}
\item \texttt{call.sh} : script utile à l'invocation du programme de manière automatisée
\item \texttt{compare.sh} : script utile à la comparaison des solutions retournées après exécution du programme
\end{enumerate}
\paragraph{}
Ces deux scripts sont disponibles dans le dossier \texttt{scripts/} fourni avec ce rapport.

\subsection{Automatisation de l'invocation de l'exécutable sur plusieurs instances}

\paragraph{}
Le script \texttt{call.sh} prend en entrée les paramètres suivants :
\begin{enumerate}
\item le chemin vers le fichier exécutable correspondant au programme ;
\item le chemin vers un dossier \texttt{instances/} contenant des instances ;
\item le chemin vers un dossier \texttt{solutions/} (existant ou à créer) qui contiendra les solutions ;
\item le nombre de mutations à effectuer à chaque itération (optionnel).
\end{enumerate}
\paragraph{}
Le script invoque automatiquement le programme sur toutes les instances du dossier \texttt{instances/} et les écrit dans le dossier \texttt{solutions/}.

\subsection{Automatisation de la comparaison des résultats obtenus}

\paragraph{}
Le script \texttt{compare.sh} prend en entrée les paramètres suivants :
\begin{enumerate}
\item le chemin vers un dossier \texttt{solutions\_1/} contenant des solutions ;
\item le chemin vers un dossier \texttt{solutions\_2/} contenant des solutions ;
\end{enumerate}
\paragraph{}
Le script recherche tous les couples de fichiers de solutions associés à une même instance :
\begin{itemize}
\item l'un des deux fichiers est situé dans le dossier \texttt{solutions\_1/} ;
\item l'autre fichier est situé dans le dossier \texttt{solutions\_2/} ;
\item les deux fichiers portent le même nom \texttt{name.txt}.
\end{itemize}
Si les deux fichiers sont différents, \texttt{name.txt} est affiché à l'écran.
\paragraph{}
À la fin, le programme affiche le nombre total de couples trouvés, le nombre de couples constitués de deux solutions différentes,
avec leur proportion (en pourcentage).

% ####################################################################################################

\section{Résultats de l'heuristique}

\paragraph{}
Nous avons utilisé le script \texttt{call.sh} sur toutes les instances dont nous disposons, regroupées dans le dossier \texttt{files/instances/}.
Dans le dossier \texttt{files/solutions/} fourni avec ce rapport, les solutions obtenues avec différentes valeurs du paramètre \og nombre de mutations \fg{} pour toutes les instances sont fournies. Ces solutions sont regroupées en sous-dossiers de la manière suivante :
\begin{center}
\texttt{S\_(valeur du paramètre : nombre de mutations)\_(lettre de \emph{a} à \emph{d})}
\end{center}
\begin{itemize}
\item Si la valeur du paramètre \og nombre de mutations \fg{} est vide, alors le script a été appelé sans préciser de valeur :
pour chaque instance, le nombre de mutations par itération est égal au nombre de jobs dans l'instance.
\item Les lettres servent à différencier plusieurs appels au script avec la même valeur pour le paramètre \og nombre de mutations \fg{}.
\end{itemize}

\paragraph{}
Des fichiers textes de comparaison des solutions obtenues sont également disponibles dans le dossier \texttt{files/solutions/compare/}.
Leur contenu est une copie de ce qu'affiche le script \texttt{compare.sh}.
Leur nom est de la forme suivante :
\begin{center}
\texttt{compare\_(nom dossier solutions 1)\_(nom dossier solutions 2).txt}
\end{center}

\subsection{Variabilité des solutions générées}

\paragraph{}
L'heuristique implémentée est extrêmement instable : en appelant deux fois de suite le programme sur la même instance, les probabilités
d'obtenir deux solutions différentes sont élevées. Ceci est dû aux choix aléatoires de solutions dans le voisinage de la solution mère
dans l'algorithme de descente locale.

\paragraph{}
Pour plus de détails, veuillez consulter les données des fichiers de comparaison, pour une même valeur du paramètre \og nombre de mutations \fg{}
et des lettres différentes.

\subsection{Influence du paramètre : nombre de mutations}

\paragraph{}
Si on augmente la valeur du paramètre \og nombre de mutations \fg{} :
\begin{enumerate}
\item Les chances d'obtenir des solutions proches d'une solution optimale sont de plus en plus grandes. (La valeur de la somme des retards diminue.)
\item Le temps d'exécution augmente :
    \begin{itemize}
    \item raisonnablement pour des instances avec peu de jobs (jusqu'à 100 jobs, moins de 5 secondes environ) ;
    \item déraisonnablement pour des instances avec beaucoup de jobs (au-delà de 150 jobs, 1 à 10 minutes environ).
    \end{itemize}
\item Le phénomène d'instabilité est accru.
\end{enumerate}

\paragraph{}
Pour plus de détails, veuillez consulter les données des fichiers de comparaison, pour des valeurs différentes du paramètre \og nombre de mutations \fg{} (quelles que soient les lettres).

% ####################################################################################################

\section{Comparaison des résultats par rapport à une autre heuristique}

\#\#\# À FAIRE !!! \#\#\#

\chapter{Conclusion}

% Le texte de la conclusion


\annexes

\end{document}
